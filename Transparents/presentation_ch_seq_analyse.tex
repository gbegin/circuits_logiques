% Created 2022-11-16 mer 14:33
% Intended LaTeX compiler: pdflatex
\documentclass[presentation]{beamer}
\usepackage[utf8]{inputenc}
\usepackage[T1]{fontenc}
\usepackage{graphicx}
\usepackage{longtable}
\usepackage{wrapfig}
\usepackage{rotating}
\usepackage[normalem]{ulem}
\usepackage{amsmath}
\usepackage{amssymb}
\usepackage{capt-of}
\usepackage{hyperref}
\usepackage{minted}
\usepackage[, french]{babel}
\usepackage{svg}
\logo{\includegraphics[width=.1\textwidth]{../../by-sa.png}}
\usetheme{metropolis}
\usecolortheme{}
\usefonttheme{}
\useinnertheme{}
\useoutertheme{}
\author{Guy Bégin}
\date{\today}
\title{Circuits logiques combinatoires et séquentiels}

\hypersetup{
 pdfauthor={Guy Bégin},
 pdftitle={Circuits logiques combinatoires et séquentiels},
 pdfkeywords={},
 pdfsubject={},
 pdfcreator={Emacs 28.1 (Org mode 9.5.4)}, 
 pdflang={French}}
\begin{document}

\maketitle

\section{Analyse de circuits logiques séquentiels synchrones}
\label{sec:orga4a65c0}
\begin{frame}[label={sec:org6862807}]{Objectifs}
\begin{itemize}
\item Pouvoir analyser en détail le comportement d'un circuit séquentiel
synchrone à partir de son schéma, selon différents types de bascules
employées
\item Pouvoir établir les équations de transition qui commandent les bascules
\item Être familier avec la notion de tableau d'excitation
\item Pouvoir tracer un diagramme d'état
\item Pouvoir interpréter le comportement du système à partir de son
diagramme d'état
\item Savoir distinguer entre les deux modèles de circuit séquentiels
(Moore et Mealy)
\end{itemize}
\end{frame}

\begin{frame}[label={sec:orge4ca596}]{Démarche d'analyse}
\begin{itemize}
\item Analyser un circuit logique séquentiel a pour but de déterminer le comportement qu'aura le circuit selon les séquences d'entrées qui lui seront appliquées et l'état dans lequel il se trouve initialement.

\item On voudra aussi connaître quelles séquences de sorties seront produites.

\item Dans la mesure où un circuit comporte une ou des bascules (peu importe le type) et un signal d'horloge, on peut considérer qu'il s'agit d'un circuit séquentiel synchrone.

\item Le type de bascule sera pris en compte pour l'analyse, qui consistera à déterminer, pour un état présent donné, quel seront les prochains états possibles selon les valeurs d'entrée.
\end{itemize}
\end{frame}

\begin{frame}[label={sec:orgfb9cb1f}]{Étapes}
Les grandes lignes de la démarche sont les suivantes.
\end{frame}

\begin{frame}[label={sec:org2cc9be0}]{Identification des éléments fonctionnels}
\begin{enumerate}
\item Identification des éléments fonctionnels:

\begin{enumerate}
\item entrées externes

\item éléments de mémoire

\item décodeur de prochain état

\item sorties externes

\item décodeur de sortie
\end{enumerate}
\end{enumerate}
\end{frame}

\begin{frame}[label={sec:org4eff5ab}]{Expressions logiques}
\begin{enumerate}
\setcounter{enumi}{1}
\item Expressions logiques du décodeur de prochain état: établies pour
chaque entrée des bascules, en fonction des entrées externes et des
variables d'état.

\item Expressions logiques des sorties externes, établies en fonction des
entrées externes et des variables d'état.
\end{enumerate}
\end{frame}

\begin{frame}[label={sec:org0b5f7b8}]{Tableau d'excitation, diagramme d'état et interprétation}
\begin{enumerate}
\setcounter{enumi}{3}
\item Construction du tableau d'excitation.

\setcounter{enumi}{4}
\item Diagramme d'état.

\item Interprétation du comportement du circuit séquentiel.
\end{enumerate}
\end{frame}

\begin{frame}[label={sec:org8626f43}]{Équations de transition}
\begin{itemize}
\item Au centre de ce processus se trouve l'analyse des circuits combinatoires qui déterminent ce que seront les entrées des bascules.

\item Nous chercherons à établir les \alert{équations de transition} qui précisent ce que sera le prochain état en fonction des entrées et de l'état présent.
\end{itemize}
\end{frame}

\begin{frame}[label={sec:org7a6e0fc}]{Exemple d'analyse}
Nous allons appliquer la démarche à un exemple qui nous permettra de
mieux expliquer chacune des étapes. Considérons le circuit de la
figure \ref{fig:orgb45f17a}.

\begin{figure}[htbp]
\centering
\includesvg[scale=0.55]{../../Sources_images_logiques/images/exemple_seq1}
\caption{\label{fig:orgb45f17a}Circuit séquentiel synchrone à analyser}
\end{figure}
\end{frame}


\begin{frame}[label={sec:orgdf61e8e}]{Identification des éléments fonctionnels}
\begin{enumerate}
\item Il y a une seule entrée externe \(I\).
\item Il y a deux éléments de mémoire, étiquetés \(Z_0^n\) et  \(Z_1^n\).
\item On a un décodeur de prochain état pour deux bascules D.
\item Il y a deux sorties externes: \(S\) et \(U\).
\item Il y a un décodeur de sortie.
\end{enumerate}
\end{frame}

\begin{frame}[label={sec:org82637d0}]{Expressions logiques pour le décodeur de prochain état}
$$ Z_0^{n+1} = I \cdot (Z_1^n)^\prime  + I \cdot Z_0^n $$

$$ Z_1^{n+1} = I \cdot Z_1^n +  Z_0^n  $$
\end{frame}

\begin{frame}[label={sec:org51c5f8c}]{Expressions logiques des sorties externes}
$$ S =  Z_0^n \cdot (Z_1^{n})^\prime $$

$$ U =  Z_0^n \cdot Z_1^{n} $$
\end{frame}

\begin{frame}[label={sec:org3ccd08a}]{Tableau d'excitation}
Chaque ligne du tableau d'excitation \ref{tab:orgab30ac7} montre, à gauche,
un état présent (identifiable par les valeurs de bascules) et une
combinaison de valeurs d'entrées, et à droite, le prochain état et les
valeurs de sortie produites.
\end{frame}

\begin{frame}[label={sec:org163c394}]{Tableau d'excitation \ldots{} 2}
\begin{table}[htbp]
\caption{\label{tab:orgab30ac7}Tableau d'excitation pour l'exemple}
\centering
\begin{tabular}{rrrlrrrr}
\(Z_1^n\) & \(Z_0^n\) & \(I\) &  & \(Z_1^{n+1}\) & \(Z_0^{n+1}\) & \(S\) & \(U\)\\
\hline
0 & 0 & 0 &  & 0 & 0 & 0 & 0\\
0 & 0 & 1 &  & 0 & 1 & 0 & 0\\
0 & 1 & 0 &  & 1 & 0 & 1 & 0\\
0 & 1 & 1 &  & 1 & 1 & 1 & 0\\
1 & 0 & 0 &  & 0 & 0 & 0 & 0\\
1 & 0 & 1 &  & 1 & 0 & 0 & 0\\
1 & 1 & 0 &  & 1 & 0 & 0 & 1\\
1 & 1 & 1 &  & 1 & 1 & 0 & 1\\
\end{tabular}
\end{table}
\end{frame}

\begin{frame}[label={sec:org9e10d8d}]{Diagramme d'état}
\begin{itemize}
\item Le diagramme d'état représente de façon schématique le comportement du circuit séquentiel.

\item Les cercles correspondent aux différents états dans lesquels le système peut se trouver.

\item On peut utiliser des étiquettes à l'intérieur des cercles pour nommer les états.

\item On peut aussi y indiquer les numéros d'états (soit en vecteurs de bits, soit avec des nombres binaires comme ici sur la figure).
\end{itemize}
\end{frame}

\begin{frame}[label={sec:orgb4bc0e3}]{Diagramme d'état \ldots{} 2}
\begin{itemize}
\item Les sorties produites par le système sont aussi indiquées dans les cercles.

\item Les transitions entre les états sont indiquées par des flèches.

\item Une flèche qui revient vers le même cercle signifie que l'état ne change pas.

\item Les conditions appliquées aux entrées pour déclencher les transitions sont indiquées sur les flèches.
\end{itemize}
\end{frame}

\begin{frame}[label={sec:orgfdc2e8b}]{Diagramme d'état \ldots{} 3}
\begin{itemize}
\item Dans la figure, on voit aussi l'état initial, identifié par une flèche partant d'un gros point noir.

\item Le diagramme d'état se construit aisément à partir du tableau d'excitation.
\end{itemize}
\end{frame}

\begin{frame}[label={sec:org416573c}]{Diagramme d'état obtenu}
\begin{figure}[htbp]
\centering
\includesvg[scale=0.55]{../../Sources_images_logiques/images/exemple_seq1_fsm}
\caption{\label{fig:org70e9402}Diagramme d'état}
\end{figure}
\end{frame}

\begin{frame}[label={sec:org7171914}]{Interprétation du comportement du circuit séquentiel.}
\begin{itemize}
\item Le système est initialement à l'état \emph{a}.

\item Tant que l'entrée \(I\) est 0, il demeure dans cet état.

\item Lorsque \(I=1\), on passe à l'état \emph{b}.

\item Si \(I\) reste à 1, on passe à l'état \emph{d} et on boucle sur l'état \emph{d}.

\item Si \(I\) revient à 0, de l'état \emph{b} ou \emph{d}, on passe à l'état \emph{c}.

\item De l'état \emph{c}, si \(I = 1\), on reste dans l'état \emph{c}.

\item Sinon (\(I = 0\)), on retourne à l'état \emph{a}.
\end{itemize}
\end{frame}

\begin{frame}[label={sec:orgfe2df44}]{Comportement}
\begin{itemize}
\item La figure \ref{fig:org8633e5a} montre une trace des formes d'onde observées en fonctionnement pour le système.

\item Dans cet exemple, le système boucle d'abord sur l'état \emph{a} (valeur 0 sur la trace) puis, passe à l'état \emph{b} (valeur 1) et ensuite l'état \emph{d} (valeur 3), boucle sur l'état \emph{d} et passe ensuite à l'état \emph{c} (valeur 2) pour finalement revenir à l'état \emph{a}.
\end{itemize}
\end{frame}

\begin{frame}[label={sec:org73bc1c0}]{Exemple de fonctionnement}
\begin{figure}[htbp]
\centering
\includesvg[scale=0.55]{../../Sources_images_logiques/images/exemple_seq1_run}
\caption{\label{fig:org8633e5a}Exemple de fonctionnement}
\end{figure}
\end{frame}

\begin{frame}[label={sec:org3690dbf}]{Analyse pour des bascules JK}
\begin{itemize}
\item Pour analyser un circuit séquentiel utilisant des bascules JK, on détermine d'abord les expressions \(J_A\) et \(K_A\), \(J_B\) et \(K_B\), etc., pour chacune des bascules.

\item On doit ensuite se référer au tableau caractéristique du précédent chapitre pour ce type de bascule pour déterminer quelles seront les prochaines valeurs de sortie pour chacune des bascules.

\item L'exemple suivant illustre la procédure.
\end{itemize}
\end{frame}

\begin{frame}[label={sec:orgca53c92}]{Exemple avec bascules JK}
Considérons le circuit séquentiel de la figure \ref{fig:org61aaa93}.

\begin{figure}[htbp]
\centering
\includesvg[scale=0.75]{../../Sources_images_logiques/images/seq_JK}
\caption{\label{fig:org61aaa93}Exemple de circuit séquentiel avec des bascules JK}
\end{figure}
\end{frame}

\begin{frame}[label={sec:orgcbf2e40}]{Exemple avec bascules JK \ldots{} 2}
À partir des expression des entrées \(J\) et \(K\) suivantes:

$$ J_0 = (q_1^{n})^\prime $$

$$ K_0 = q_0^{n} \cdot (q_1^{n})^\prime $$

$$ J_1 = q_0^{n} $$

$$ K_1 = (q_0^{n})^\prime \cdot q_1^{n} $$

on peut remplir le tableau d'excitation (tableau \ref{tab:org6166e18}).
\end{frame}

\begin{frame}[label={sec:orga7c25cf}]{Tableau d'excitation circuit séquentiel JK}
\begin{table}[htbp]
\caption{\label{tab:org6166e18}Tableau d'excitation circuit séquentiel JK}
\centering
\begin{tabular}{rrrrrrlrr}
\(q_1^{n+1}\) & \(q_0^n\) & \(J_0\) & \(K_0\) & \(J_1\) & \(K_1\) &  & \(q_1^{n+1}\) & \(q_0^{n+1}\)\\
\hline
0 & 0 & 1 & 0 & 0 & 0 &  & 0 & 1\\
0 & 1 & 1 & 1 & 1 & 0 &  & 1 & 0\\
1 & 0 & 0 & 0 & 0 & 1 &  & 0 & 0\\
1 & 1 & 0 & 0 & 1 & 0 &  & 1 & 1\\
\end{tabular}
\end{table}
\end{frame}

\begin{frame}[label={sec:orgd4e853a}]{Diagramme d'état}
À partir du tableau d'excitation, on peut tracer le diagramme d'état
(figure \ref{fig:org3cd5540}).

\begin{figure}[htbp]
\centering
\includesvg[scale=0.55]{../../Sources_images_logiques/images/seq_JKb_FSM}
\caption{\label{fig:org3cd5540}Diagramme d'état du circuit séquentiel avec bascules JK}
\end{figure}
\end{frame}

\begin{frame}[label={sec:org0063b9d}]{Modèles de machines séquentielles}
\begin{itemize}
\item On appelle les modèles abstraits de systèmes séquentiels des \alert{machines à état fini} (en anglais, \emph{Finite State Machines}, (FSM)).

\item On distingue deux modèles de circuit séquentiels, selon la façon dont les sortie sont obtenues.

\item Dans le modèle de Mealy, les sorties dépendent à la fois des entrées et des variables d'état présent (figure \ref{fig:org7fbfceb}).

\item Dans le modèle de Moore, les sorties ne dépendent que des variables d'état présent (figure \ref{fig:orgaa31e29}).
\end{itemize}
\end{frame}

\begin{frame}[label={sec:org26e23c5}]{Machine de Mealy}
\begin{figure}[htbp]
\centering
\includegraphics[width=.9\linewidth]{../../Sources_images_logiques/images/mealy.png}
\caption{\label{fig:org7fbfceb}Machine de Mealy}
\end{figure}
\end{frame}

\begin{frame}[label={sec:org21e27ea}]{Machine de Moore}
\begin{figure}[htbp]
\centering
\includegraphics[width=.9\linewidth]{../../Sources_images_logiques/images/moore.png}
\caption{\label{fig:orgaa31e29}Machine de Moore}
\end{figure}
\end{frame}
\end{document}