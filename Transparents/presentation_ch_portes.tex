% Created 2023-10-26 jeu 09:51
% Intended LaTeX compiler: pdflatex
\documentclass[presentation]{beamer}
\usepackage[utf8]{inputenc}
\usepackage[T1]{fontenc}
\usepackage{graphicx}
\usepackage{longtable}
\usepackage{wrapfig}
\usepackage{rotating}
\usepackage[normalem]{ulem}
\usepackage{amsmath}
\usepackage{amssymb}
\usepackage{capt-of}
\usepackage{hyperref}
\usepackage{minted}
\usepackage[, french]{babel}
\usepackage{svg}
\logo{\includegraphics[width=.1\textwidth]{../by-sa.png}}
\AtBeginEnvironment{minted}{\renewcommand{\fcolorbox}[4][]{#4}}
\usetheme{metropolis}
\usecolortheme{}
\usefonttheme{}
\useinnertheme{}
\useoutertheme{}
\author{Guy Bégin}
\date{\today}
\title{Circuits logiques combinatoires et séquentiels}

\hypersetup{
 pdfauthor={Guy Bégin},
 pdftitle={Circuits logiques combinatoires et séquentiels},
 pdfkeywords={},
 pdfsubject={},
 pdfcreator={Emacs 28.1 (Org mode 9.6.6)}, 
 pdflang={French}}
\begin{document}

\maketitle

\section{Portes logiques}
\label{sec:orgec4e083}

\begin{frame}[label={sec:orga00e654}]{Objectifs}
\begin{itemize}
\item Se familiariser avec les symboles usuels des portes logiques
\item Se familiariser avec les conventions et règles de dessin de schémas
logiques
\item Faire la différence entre niveau de signal et valeur logique
\item Expliquer les différences entre le fonctionnement idéalisé
et la réalité physique des portes logiques
\end{itemize}
\end{frame}

\begin{frame}[label={sec:org98896f4}]{Niveaux logiques}
\begin{itemize}
\item Une porte logique est un dispositif électronique qui implémente une fonction logique en agissant sur des signaux électriques selon une convention préétablie.

\item En général, on établit des valeurs binaires en se basant sur la tension des signaux, en définissant une correspondance entre des gammes de tensions et les valeurs logiques 0 et 1.

\item Par exemple, pour une tension d'alimentation \(V_{DD}\), on pourrait avoir les correspondances suivantes:
\end{itemize}

\begin{center}
\begin{tabular}{ll}
Gamme de tensions & Niveau\\[0pt]
\hline
de 0 à  \(V_{DD}/3\) & Niveau bas\\[0pt]
de \(2V_{DD}/3\) à  \(V_{DD}\) & Niveau haut\\[0pt]
\end{tabular}
\end{center}
\end{frame}

\begin{frame}[label={sec:org2af7373}]{Niveaux logiques \ldots{} 2}
\begin{itemize}
\item Les portes logiques sont manufacturées selon différents standards technologiques qu'on appelle communément des \alert{familles logiques}.

\item Au sein d'une même famille, les portes respectent les mêmes références de niveaux pour pouvoir fonctionner ensemble adéquatement.

\item Une porte peut comporter une ou plusieurs entrées et agit généralement sur une seule sortie.
\end{itemize}
\end{frame}

\begin{frame}[label={sec:org693a84d}]{Logique négative ou positive}
\begin{itemize}
\item On associe ensuite une valeur binaire à chacun des niveaux selon une certaine convention, par exemple:
\end{itemize}
\begin{center}
\begin{tabular}{lr}
Niveau & Valeur logique\\[0pt]
\hline
Niveau bas & 0\\[0pt]
Niveau haut & 1\\[0pt]
\end{tabular}
\end{center}
qui correspond à une logique positive. La convention inverse nous donne la logique négative.
\end{frame}

\begin{frame}[label={sec:org58133bb}]{Logique négative ou positive \ldots{} 2}
Certains signaux seront considérés comme actifs lorsque leur niveau logique sera 0.

On parlera alors de signaux \alert{actifs bas}.

La convention implicite est généralement que les signaux sont \alert{actifs haut}.
\end{frame}

\begin{frame}[label={sec:org968ae7e}]{Symboles}
\begin{itemize}
\item On a défini des symboles pour représenter graphiquement les portes logiques courantes.

\item Dans un schéma logique, les portes sont interconnectées entre-elles au moyen de symboles de conducteurs (fils) qui permettent d'acheminer les valeurs logiques d'une porte à l'autre.
\end{itemize}
\end{frame}

\begin{frame}[label={sec:org064f4d8}]{Porte ET}
À deux entrées \(S =  A \cdot B\)

\begin{figure}[htbp]
\centering
\includesvg[scale=0.75]{../Images_svg/and_logique}
\caption{\label{fig:orgf8b8792}Porte ET à deux entrées}
\end{figure}

\begin{itemize}
\item Les portes qui réalisent des fonctions qui sont associatives et commutatives peuvent aussi se définir avec plus de deux entrées.

\item C'est le cas avec les fonctions ET et OU.
\end{itemize}
\end{frame}

\begin{frame}[label={sec:org5f81a41}]{Porte ET}
À trois entrées \(S =  A \cdot B \cdot C\)

\begin{figure}[htbp]
\centering
\includesvg[scale=0.75]{../Images_svg/and3_logique}
\caption{\label{fig:org7f42344}Porte ET à trois entrées}
\end{figure}
\end{frame}

\begin{frame}[label={sec:org149cbff}]{Porte OU}
À deux entrées \(S =  A + B\)

\begin{figure}[htbp]
\centering
\includesvg[scale=0.75]{../Images_svg/or_logique}
\caption{\label{fig:org5b7917f}Porte OU à deux entrées}
\end{figure}
\end{frame}

\begin{frame}[label={sec:org3509f3c}]{Porte inverseur}
\begin{itemize}
\item L'opération NON qui consiste à complémenter une valeur binaire s'effectue avec une porte appelée \alert{inverseur}.

\item Il n'y a toujours qu'une entrée. \(B = A^\prime\)
\end{itemize}

\begin{figure}[htbp]
\centering
\includesvg[scale=0.75]{../Images_svg/not_logique}
\caption{\label{fig:org489ba29}Porte inverseur}
\end{figure} 
\end{frame}

\begin{frame}[label={sec:org7f1d1b8}]{Porte NON-OU (NOR)}
\begin{figure}[htbp]
\centering
\includesvg[scale=0.75]{../Images_svg/nor_logique}
\caption{\label{fig:orgf7fd24d}Porte NOR à deux entrées}
\end{figure}
\end{frame}

\begin{frame}[label={sec:org32a02cf}]{Porte NON-ET (NAND) et NON\textsubscript{OU} (NOR)}
Les fonctions NAND et NOR ne sont pas associatives. Par exemple,

$$
(x \operatorname{Nor} y) \operatorname{Nor} z \neq x \operatorname{Nor} (y \operatorname{Nor} z) 
$$

\begin{itemize}
\item On peut néanmoins définir des versions à plusieurs entrées de ces fonctions en ajustant la priorité d'évaluation.
\end{itemize}
\end{frame}

\begin{frame}[label={sec:org638783d}]{Porte NON-ET (NAND) et NON-OU (NOR)}
\begin{itemize}
\item Pour une porte NAND à trois entrées, on fera \(S = (A \cdot B \cdot C)^\prime\).
\end{itemize}

Pour une porte NOR à trois entrées, on fera \((A + B + C)^\prime\).



\begin{itemize}
\item Pour une porte NAND à trois entrées, on fera \(S = (A \cdot B \cdot C)^\prime\).
\end{itemize}

\begin{figure}[htbp]
\centering
\includesvg[scale=0.75]{../Images_svg/nand3_logique}
\caption{\label{fig:org73916ae}Porte NAND à trois entrées}
\end{figure}
\end{frame}

\begin{frame}[label={sec:orge39c8d5}]{Entrées inversées}
\begin{itemize}
\item On utilise souvent l'élément symbolique qui est placé à la sortie de l'inverseur (un petit cercle) pour indiquer l'inversion d'une entrée ou d'une sortie d'une porte.

\item C'est le cas à la sortie des portes NAND et NOR comme on vient de le voir.

\item Un autre exemple est la porte NAND de la figure \ref{fig:org8edaf9f}, où une des entrées est également inversée.

\item La porte évalue donc \(S = (A^\prime \cdot B \cdot C)^\prime\)
\end{itemize}

\begin{figure}[htbp]
\centering
\includesvg[scale=0.75]{../Images_svg/nand3_logique_invin1}
\caption{\label{fig:org8edaf9f}Porte NAND à trois entrées dont une inversée}
\end{figure} 
\end{frame}

\begin{frame}[label={sec:orgfa1fb1e}]{NAND et NOR, représentations équivalentes}
\begin{itemize}
\item En vertu du théorème de DeMorgan, on sait que \((x + y)^{\prime} = x^{\prime} y^{\prime}\) et que \((xy)^{\prime} = x^{\prime} + y^{\prime}\).

\item On peut donc représenter les portes NAND et NOR de deux façons équivalentes.
\end{itemize}

\begin{figure}[htbp]
\centering
\includesvg[scale=0.75]{../Images_svg/NANDequiv}
\caption{\label{fig:orgbe0f568}Deux représentations équivalentes pour une porte NAND}
\end{figure}

\begin{figure}[htbp]
\centering
\includesvg[scale=0.75]{../Images_svg/NORequiv}
\caption{\label{fig:orga7fa8b3}Deux représentations équivalentes pour une porte NOR}
\end{figure}
\end{frame}


\begin{frame}[label={sec:orgd315d45}]{Porte OU-exclusif (XOR)}
\begin{itemize}
\item La porte XOR à deux entrées donne une sortie 1 seulement lorsque ses deux entrées sont différentes.

\item Il est possible de définir des portes XOR à plus de deux entrées, mais il y a différentes interprétations de ce qu'une telle porte devrait avoir comme comportement.

\item De plus, comme la réalisation pratique de cette fonction n'est pas aussi simple que pour les autres fonctions, on se retrouve la plupart du temps à devoir mettre des portes à deux entrées en cascade pour augmenter le nombre d'entrées, ce qui rend moins intéressantes les portes XOR avec entrées nombreuses.
\end{itemize}
\end{frame}

\begin{frame}[label={sec:orga7e3cd6}]{Porte OU-exclusif (XOR) \ldots{} 2}
$$ S= A \cdot B^\prime + A^\prime \cdot B $$  

\begin{figure}[htbp]
\centering
\includesvg[scale=0.75]{../Images_svg/exor_logique}
\caption{\label{fig:org2c56235}Porte XOR à deux entrées}
\end{figure}
\end{frame}

\begin{frame}[label={sec:org9235ca3}]{Porte NON-OU-exclusif ou Équivalence (XNOR)}
\begin{itemize}
\item La porte \alert{Équivalence} produit une sortie 1 lorsque ses entrées ont la même valeur (et sont donc équivalentes).

\item Comme pour les portes XOR, les portes XNOR à plus de trois entrées peuvent s'interpréter de différentes façons.
\end{itemize}

\begin{figure}[htbp]
\centering
\includesvg[scale=0.75]{../Images_svg/xnor_logique}
\caption{\label{fig:org423c81c}Porte XNOR}
\end{figure}
\end{frame}

\begin{frame}[label={sec:org1d3c46e}]{Universalité des NAND et NOR}
En faisant appel uniquement à des portes de type NAND ou NOR, il est possible de réaliser n'importe quelle fonction logique, puisqu'il est possible de réaliser les trois opérateurs de base.

\begin{enumerate}
\item Pour réaliser un inverseur, on utilise une porte NAND à une seule entrée (ou dont toutes les entrées sont reliées ensemble).
\item Pour réaliser une porte ET, on fait suivre une porte NAND d'un inverseur.
\item Pour réaliser une porte OU, on place un inverseur devant chaque entrée d'une porte NAND.
\end{enumerate}

Nous verrons plus loin qu'il est aussi possible de réaliser avantageusement des fonctions quelconques avec des portes NAND en exploitant la forme \emph{somme de produits}.
\end{frame}

\begin{frame}[label={sec:orgff7186a}]{Limites physiques}
\begin{itemize}
\item Les portes logiques qu'on utilisera en pratique sont des dispositifs électroniques dont le fonctionnement correspond, dans les grandes lignes, aux comportements idéalisés des modèles abstraits de l'algèbre de Boole.

\item Mais il faut toujours garder à l'esprit que la correspondance entre modèle et réalité physique n'est jamais parfaite.

\item En raffinant nos modèles pour y incorporer des caractéristiques, limites ou contraintes appropriées, il sera possible de mieux tenir compte de la réalité physique.
\end{itemize}
\end{frame}

\begin{frame}[label={sec:orgb2fcacf}]{Sortance (\emph{Fan-out})}
\begin{itemize}
\item La sortance (\emph{fan-out}) d'une porte logique mesure sa capacité à commander d'autres portes reliées à sa sortie.

\item Puisque les portes sont des dispositifs électroniques qui doivent faire circuler un certain courant électrique pour concrétiser les niveaux de tensions qui définissent leurs valeurs d'entrée et de sortie, il y a une limite pratique à la capacité d'une porte de fournir le courant nécessaire pour faire réagir la sortie des portes qu'elle devrait commander.

\item La sortance mesure cette limite, en nombre de portes à commander.

\item Si on connecte plus d'entrées à une sortie que sa valeur de sortance, cette sortie ne pourra pas atteindre le niveau de tension adéquat, et les opérations logiques seront faussées.
\end{itemize}
\end{frame}

\begin{frame}[label={sec:orgc075112}]{Modèles de délai}
\begin{itemize}
\item Dans la mesure où on respecte ses contraintes d'utilisation, notamment de sortance, une porte logique se comporte globalement de la façon attendue, étant donné sa fonction et les conventions de niveaux de signal établies.

\item Par exemple, le niveau signal à la sortie d'un inverseur correspondra au niveau de signal attendu pour le complément de la valeur logique à son entrée.

\item Mais il faut garder à l'esprit que les portes sont des dispositifs électroniques, et donc physiques, sujets à des «imperfections» qui diffèrent du comportement idéalisé.
\end{itemize}
\end{frame}

\begin{frame}[label={sec:org0ca5e26}]{Modèles de délai \ldots{} 2}
\begin{itemize}
\item Une de ces «imperfections» dont on doit impérativement tenir compte est le \alert{délai de propagation} qui se manifeste comme un retard entre le moment où le signal à l'entrée de la porte assume (se stabilise à) son niveau de signal et le moment où la sortie de la porte atteint son niveau de signal attendu.

\item C'est en quelque sorte le délai entre une action à l'entrée et son effet sur la sortie.

\item Ce délai limite la vitesse à laquelle on peut utiliser notre circuit logique.

\item Si on essaie d'effectuer des transitions plus rapides que le délai, le comportement ne sera plus conforme aux attentes de conception.

\item On doit donc respecter une vitesse de commutation maximale imposée par les délais de propagation.
\end{itemize}
\end{frame}

\begin{frame}[label={sec:orgcb645c0}]{Modèles de délai \ldots{} 3}
\begin{itemize}
\item Le délai de propagation peut dépendre de plusieurs facteurs: la famille logique, le type de porte, le sens de la transition, la
\end{itemize}
sortance effective, les caractéristiques d'interconnexions, etc.

\begin{itemize}
\item Pour faciliter l'analyse, on fait appel à des modèles de délais plus ou moins sophistiqués.

\item Un modèle très simple consiste à supposer un délai de propagation moyen, constant pour toutes les portes d'une famille donnée.

\item Un modèle un peu plus subtil pourrait prendre en compte des délais de propagation moyens différents par types de portes.

\item Le délai de propagation moyen est une caractéristique clé qui différencie les différentes familles logiques.

\item Les délais sont typiquement de l'ordre de nanosecondes, permettant des vitesses de commutation dans les dizaines, centaines, voire des milliers de MHz.
\end{itemize}
\end{frame}

\begin{frame}[label={sec:org55a55a3}]{Modèles de délai \ldots{} 4}
\begin{itemize}
\item Lorsqu'un signal doit se propager à travers plusieurs portes, les délais de propagation s'accumulent, limitant encore davantage la vitesse de commutation de l'ensemble du circuit.

\item La vitesse qui pourra être atteinte pour l'ensemble d'un circuit sera typiquement déterminée par le plus lent chemin (c'est-à-dire celui qui cumule le plus long temps de propagation).
\end{itemize}
\end{frame}

\begin{frame}[label={sec:org14ff7fb}]{Modèles simples}
\begin{itemize}
\item À titre d'exemple, considérons une porte ET à deux entrées \(S = A B\).

\item Le modèle le plus simple suppose une porte idéale, sans aucun délai: le chronogramme suivant montre la sortie qui commute immédiatement lorsque les conditions d'entrée changent.
\end{itemize}

\begin{figure}[htbp]
\centering
\includesvg[width=.9\linewidth]{../Images_svg/chronopasdelais}
\caption{\label{fig:orge79a907}Porte ET sans délai}
\end{figure}
\end{frame}

\begin{frame}[label={sec:org26e29b4}]{Modèle avec délai en sortie}
\begin{itemize}
\item Le modèle avec délai en sortie consiste à considérer un délai fixe, qui affecte la sortie de la porte: la commutation prend effet en sortie après un délai \(t_p\).
\end{itemize}


\begin{figure}[htbp]
\centering
\includesvg[width=.9\linewidth]{../Images_svg/chrononodelaisortie}
\caption{\label{fig:org23533ad}Porte ET avec délai en sortie}
\end{figure}
\end{frame}

\begin{frame}[label={sec:org2e0e8bc}]{Modèle avec délai en entrée}
\begin{itemize}
\item Le modèle avec délai en entrée est plus nuancé, car il permet de spécifier un délai différent selon l'entrée qui entraîne le changement à la sortie.
\end{itemize}

\begin{figure}[htbp]
\centering
\includesvg[width=.9\linewidth]{../Images_svg/chrononodelaientree}
\caption{\label{fig:orgfcbb184}Porte ET avec délai aux entrées}
\end{figure}
\end{frame}

\begin{frame}[label={sec:org631259d}]{Modèle combiné}
\begin{itemize}
\item Le modèle combiné consiste à considérer des délais différents par entrée et, en plus, un délai global en sortie.
\end{itemize}
\end{frame}

\begin{frame}[label={sec:orgbdc0db0}]{Condition de course et aléas}
\begin{itemize}
\item Un autre effet néfaste potentiel des délais à considérer est ce qu'on appelle une \alert{condition de course}.

\item Considérons le circuit de la figure \ref{fig:org29df200}.

\item La sortie de la porte est \(s = a \cdot a^\prime\) qui devrait normalement donner systématiquement 0.

\item Mais le chemin menant de l'entrée \(a\) à l'entrée du haut de la porte ET est plus court (en termes de délais) que le chemin qui mène à l'entrée du bas.

\item En effet, le signal \(a^\prime\) est retardé d'un délai de propagation \(t_{p1}\) par rapport à \(a\).
\end{itemize}

\begin{figure}[htbp]
\centering
\includesvg[scale=0.75]{../Images_svg/course}
\caption{\label{fig:org29df200}Cas à risque de condition de course}
\end{figure}
\end{frame}

\begin{frame}[label={sec:org0e7e9b3}]{Condition de course et aléas \ldots{} 2}
\begin{itemize}
\item En pratique, on pourrait observer un chronogramme qui s'apparente à celui de la figure suivante (figure \ref{fig:orgbf26aa3}), où on voit que les deux signaux à l'entrée de la porte ET sont simultanément égaux à 1 pendant une courte période.
\end{itemize}


\begin{figure}[htbp]
\centering
\includesvg[width=.9\linewidth]{../Images_svg/chronocourse}
\caption{\label{fig:orgbf26aa3}Chronogramme montrant une condition de course}
\end{figure}
\end{frame}
\begin{frame}[label={sec:org054fbe0}]{Condition de course et aléas \ldots{} 3}
\begin{itemize}
\item Une courte impulsion 1 sera donc générée sur le signal \(s\) en sortie de la porte ET, après le délai de propagation \(t_{p2}\) de celle-ci.

\item Cette impulsion, qui ne correspond à rien selon la logique du circuit est appelée un \alert{aléa} (ou en anglais, \emph{glitch}).

\item Ces aléas peuvent être la source de problèmes et de dysfonctionnements qui sont parfois difficiles à diagnostiquer, et il faut vraiment s'en méfier.

\item Une telle impulsion, quasi imperceptible, pourrait par exemple déclencher le basculement de la valeur d'une cellule mémoire plus loin dans le circuit.
\end{itemize}
\end{frame}

\begin{frame}[label={sec:orgc16118c}]{Porte tampon}
\begin{itemize}
\item La valeur binaire à la sortie d'une porte tampon est la même qu'à l'entrée.

\item La porte n'agit pas sur la valeur logique mais permet de reconditionner le signal à son entrée pour le rendre, en sortie, davantage conforme aux niveaux électriques de référence.

\item Une porte tampon est essentiellement utilisée pour renforcer et stabiliser le niveau du signal.

\item Une façon pratique de réaliser une porte tampon est de placer deux inverseurs l'un à la suite de l'autre.

\item L'utilisation de portes tampons est un des moyens de s'assurer de respecter les conditions de sortance.
\end{itemize}
\end{frame}
\end{document}