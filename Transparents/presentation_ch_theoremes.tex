% Created 2022-11-16 mer 14:32
% Intended LaTeX compiler: pdflatex
\documentclass[presentation]{beamer}
\usepackage[utf8]{inputenc}
\usepackage[T1]{fontenc}
\usepackage{graphicx}
\usepackage{longtable}
\usepackage{wrapfig}
\usepackage{rotating}
\usepackage[normalem]{ulem}
\usepackage{amsmath}
\usepackage{amssymb}
\usepackage{capt-of}
\usepackage{hyperref}
\usepackage{minted}
\usepackage[, french]{babel}
\usepackage{svg}
\logo{\includegraphics[width=.1\textwidth]{../../by-sa.png}}
\usetheme{metropolis}
\usecolortheme{}
\usefonttheme{}
\useinnertheme{}
\useoutertheme{}
\author{Guy Bégin}
\date{\today}
\title{Circuits logiques combinatoires et séquentiels}

\hypersetup{
 pdfauthor={Guy Bégin},
 pdftitle={Circuits logiques combinatoires et séquentiels},
 pdfkeywords={},
 pdfsubject={},
 pdfcreator={Emacs 28.1 (Org mode 9.5.4)}, 
 pdflang={French}}
\begin{document}

\maketitle


\section{Théorèmes et propriétés}
\label{sec:orgd3769e9}
\begin{frame}[label={sec:org4e97259}]{Objectifs}
\begin{itemize}
\item Bien saisir les relations de dualité entre les opérations
\item Connaître les principaux théorèmes de l'algèbre de Boole et pouvoir
les appliquer correctement
\item Pouvoir passer d'une version d'un théorème à sa version duale
\item Connaître les autres fonctions logiques importantes
\item Pouvoir construire un tableau de vérité
\end{itemize}
\end{frame}

\begin{frame}[label={sec:org960e1d1}]{Dualité}
\begin{itemize}
\item Les postulats ont été formulés en paires, identifiés par \(\spadesuit\) et \(\heartsuit\).

\item En interchangeant les opérateurs et les éléments identité, on transforme un postulat de forme \(\spadesuit\) en un postulat de forme \(\heartsuit\).

\item C'est le principe de \alert{dualité}.

\item Ainsi, n'importe quelle expression algébrique demeurera valide si les opérateurs et les valeurs d'éléments identité sont interchangés.

\item Puisque notre algèbre ne comporte que deux éléments, les deux éléments identité sont en fait les deux seuls éléments, 0 et 1.

\item On obtient donc le dual d'une expression en changeant les 0 pour des 1, les 1 pour des 0 et les ET pour des OU, les OU pour des ET.
\end{itemize}
\end{frame}

\begin{frame}[label={sec:org22a2c4e}]{Théorèmes de base}
Le tableau \ref{tab:orgf464d1e} résume les postulats et théorèmes de base de
notre algèbre. On présente en parallèle chaque version et sa version
duale.
\end{frame}
\begin{frame}[label={sec:orgd8dbca5}]{Théorèmes de base \ldots{} 2}
\begin{table}[htbp]
\caption{\label{tab:orgf464d1e}Théorèmes de l'algèbre de Boole}
\centering
\begin{tabular}{lll}
 & Version  \(\spadesuit\) & Version  \(\heartsuit\)\\
\hline
Postulat 2 & \(x+0=x\) & \(x \cdot 1 = x\)\\
Postulat 5 & \(x+x^{\prime} = 1\) & \(x \cdot x^{\prime} = 0\)\\
Theorème 1 & \(x + x = x\) & \(x \cdot x = x\)\\
Theorème 2 & \(x + 1 = 1\) & \(x \cdot 0 = 0\)\\
Theorème 3 & \((x^{\prime})^{\prime} = x\) & \\
Postulat 3 & \(x + y = y + x\) & \(xy = yx\)\\
Theorème 4 & \(x + (y + z) = (x + y ) + z\) & \(x(yz) = (xy)z\)\\
Postulat 4 & \(x(y+z) = xy + xz\) & \(x + yz = (x+y)(x+z)\)\\
Theorème 5 & \((x + y)^{\prime} = x^{\prime} y^{\prime}\) & \((xy)^{\prime} = x^{\prime} + y^{\prime}\)\\
Theorème 6 & \(x + xy = x\) & \(x(x+y) = x\)\\
\end{tabular}
\end{table}
\end{frame}

\begin{frame}[label={sec:org7255333}]{Autres fonctions logiques}
\begin{itemize}
\item Nous avons vu que les opérateurs logiques ET, OU et NON, qu'on peut aussi appeler fonctions logiques, sont à la base même de la définition de notre algèbre de Boole.

\item Il est possible de concevoir d'autres fonctions logiques qui vont s'avérer utiles pour la formulation, la conception et la réalisation de systèmes logiques. Voici quelques unes des plus souvent utilisées.
\end{itemize}
\end{frame}

\begin{frame}[label={sec:org0b0ab54}]{Fonction NON-ET}
\begin{itemize}
\item La fonction NON-ET, souvent désignée NAND, est obtenue en complémentant la sortie d'une fonction ET: \((x \cdot y)^\prime\).
\end{itemize}

\begin{table}[htbp]
\caption{\label{tab:orgafdacc7}Tableau de vérité de la fonction NON-ET}
\centering
\begin{tabular}{rrlr}
\(x\) & \(y\) &  & \((x \cdot y)^\prime\)\\
\hline
0 & 0 &  & 1\\
0 & 1 &  & 1\\
1 & 0 &  & 1\\
1 & 1 &  & 0\\
\end{tabular}
\end{table}
\end{frame}

\begin{frame}[label={sec:org0ab3aa0}]{Fonction NON-OU (NOR)}
\begin{itemize}
\item La fonction NON-OU, souvent désignée NOR, est obtenue en complémentant la sortie d'une fonction OU: \((x + y)^\prime\).
\end{itemize}

\begin{table}[htbp]
\caption{\label{tab:org0688c2f}Tableau de vérité de la fonction NON-OU}
\centering
\begin{tabular}{rrlr}
\(x\) & \(y\) &  & \((x + y)^\prime\)\\
\hline
0 & 0 &  & 1\\
0 & 1 &  & 0\\
1 & 0 &  & 0\\
1 & 1 &  & 0\\
\end{tabular}
\end{table}
\end{frame}

\begin{frame}[label={sec:orgfdba10e}]{Fonction OU-exclusif (XOR)}
\begin{itemize}
\item La fonction OU-exclusif, souvent désignée XOR, est obtenue en évaluant \(x \cdot y^\prime + x^\prime \cdot y\).

\item La sortie est 1 seulement si une seule des entrées est 1.

\item On verra plus loin que cette fonction joue un rôle important dans la formulation d'un additionneur.
\end{itemize}

\begin{table}[htbp]
\caption{\label{tab:orgd7d98cc}Tableau de vérité de la fonction OU-exclusif}
\centering
\begin{tabular}{rrlr}
\(x\) & \(y\) &  & \((x \cdot y^\prime + x^\prime \cdot y)\)\\
\hline
0 & 0 &  & 0\\
0 & 1 &  & 1\\
1 & 0 &  & 1\\
1 & 1 &  & 0\\
\end{tabular}
\end{table}
\end{frame}

\begin{frame}[label={sec:orgd0f89ad}]{Fonctions de plusieurs entrées}
\begin{itemize}
\item La plupart des fonctions logiques simples peuvent naturellement se formuler en fonction de plus de deux entrées.

\item Par exemple, \(a \cdot b \cdot c\) nous donne une fonction ET à trois entrées, et on peut facilement imaginer des fonctions ET ou des fonctions OU avec encore plus d'entrées.
\end{itemize}
\end{frame}

\begin{frame}[label={sec:org86a0d12}]{Expressions et fonctions binaires}
\begin{itemize}
\item Une fonction binaire peut être décrite par une expression algébrique Booléenne.

\item Selon les valeurs des variables, la valeur de l'expression Booléenne détermine la valeur de la fonction.

\item Par exemple, \(F_1\) est une fonction de trois entrées \(a\) \(b\) et \(c\) définie par l'expression
\end{itemize}

$$ F_1 = a + b \cdot c^\prime $$
\end{frame}

\begin{frame}[label={sec:org854879e}]{Expressions et fonctions binaires \ldots{} 2}
\begin{itemize}
\item La précédence des opération dans les expressions algébriques est (1) parenthèses, (2) NON, (3) ET, et (4) OU.

\item Il est possible de construire le tableau de vérité pour \(F_1\) en évaluant la fonction pour les \(2^3 = 8\) combinaisons d'entrées possibles, comme dans le tableau \ref{tab:org724446f}.
\end{itemize}
\end{frame}

\begin{frame}[label={sec:org2d20f76}]{Expressions et fonctions binaires \ldots{} 3}
\begin{table}[htbp]
\caption{\label{tab:org724446f}Fonction de trois variables}
\centering
\begin{tabular}{rrrlr}
\(a\) & \(b\) & \(c\) &  & \(F_1\)\\
\hline
0 & 0 & 0 &  & 0\\
0 & 0 & 1 &  & 0\\
0 & 1 & 0 &  & 1\\
0 & 1 & 1 &  & 0\\
1 & 0 & 0 &  & 1\\
1 & 0 & 1 &  & 1\\
1 & 1 & 0 &  & 1\\
1 & 1 & 1 &  & 1\\
\end{tabular}
\end{table}

En général, pour une fonction à \(n\) entrées, le tableau de vérité
comportera \(2^n\) lignes.
\end{frame}

\begin{frame}[label={sec:org56a1515}]{Théorèmes de DeMorgan}
\begin{itemize}
\item Le complément d'une fonction \(F\), \(F^\prime\), s'obtient en remplaçant tous les 0 par des 1 et tous les 1 par des 0 dans les valeurs de la fonction.

\item Par exemple, en complémentant ainsi les valeurs dans le tableau de vérité, on effectue ce changement.

\item On peut aussi effectuer ce changement en appliquant les théorèmes de DeMorgan (Théorème 5 \(\spadesuit\) et \(\heartsuit\) du tableau \ref{tab:orgf464d1e}) qui peuvent se généraliser à plus de deux variables.
\end{itemize}
\end{frame}
\end{document}