% Created 2023-10-26 jeu 09:52
% Intended LaTeX compiler: pdflatex
\documentclass[presentation]{beamer}
\usepackage[utf8]{inputenc}
\usepackage[T1]{fontenc}
\usepackage{graphicx}
\usepackage{longtable}
\usepackage{wrapfig}
\usepackage{rotating}
\usepackage[normalem]{ulem}
\usepackage{amsmath}
\usepackage{amssymb}
\usepackage{capt-of}
\usepackage{hyperref}
\usepackage{minted}
\usepackage[, french]{babel}
\usepackage{svg}
\logo{\includegraphics[width=.1\textwidth]{../by-sa.png}}
\AtBeginEnvironment{minted}{\renewcommand{\fcolorbox}[4][]{#4}}
\usetheme{metropolis}
\usecolortheme{}
\usefonttheme{}
\useinnertheme{}
\useoutertheme{}
\author{Guy Bégin}
\date{\today}
\title{Circuits logiques combinatoires et séquentiels}

\hypersetup{
 pdfauthor={Guy Bégin},
 pdftitle={Circuits logiques combinatoires et séquentiels},
 pdfkeywords={},
 pdfsubject={},
 pdfcreator={Emacs 28.1 (Org mode 9.6.6)}, 
 pdflang={French}}
\begin{document}

\maketitle

\section{Logique binaire, fonctions logiques et algèbre de Boole}
\label{sec:org7ee3e99}


\begin{frame}[label={sec:orgbb6b8be}]{Objectifs}
\begin{itemize}
\item Situer les opérations de la logique binaire dans leur contexte algébrique
\item Se familiariser avec les postulats de l'algèbre de Boole, et les
principaux théorèmes
\item Exprimer une fonction logique par un tableau de vérité
\item Formuler une expression logique à partir d'un tableau de vérité
\item Exprimer une fonction logique en \emph{somme de produits}, ou en
\emph{produit de sommes}, et  convertir d'une forme à l'autre
\end{itemize}
\end{frame}

\begin{frame}[label={sec:orgb6ead88}]{Logique binaire}
\begin{itemize}
\item La logique binaire associe une valeur de vérité à des variables,
selon une convention préétablie.

\item Ces valeurs de vérité sont binaires, à savoir, \alert{vrai} ou \alert{faux}.

\item Pour représenter ces valeurs de vérité, on peut utiliser un encodage
binaire,  par exemple:
\end{itemize}

\begin{center}
\begin{tabular}{llr}
Valeur de vérité &  & Valeur binaire\\[0pt]
\hline
Vrai &  & 1\\[0pt]
Faux &  & 0\\[0pt]
\end{tabular}
\end{center}
\end{frame}


\begin{frame}[label={sec:orge70a30a}]{Variable binaire}
\begin{itemize}
\item Une variable binaire, dénotée par une lettre, permet de désigner une valeur binaire pouvant assumer une des deux valeurs possible, 0 ou 1.

\item La variable est typiquement associée à une proposition, à l'état d'un élément ou à toute autre condition pouvant admettre deux états distincts.

\item En assignant une valeur binaire à la variable, on définit une valeur de vérité associée à cette variable, et ainsi à la condition qu'elle représente.

\item Par exemple, soit \(S\) une variable binaire qui représente la proposition «le soleil est visible».

\item Alors, \(S=0\) peut s'interpréter comme «le soleil est visible est faux» ou «le soleil n'est pas visible».
\end{itemize}
\end{frame}

\begin{frame}[label={sec:org2e5c247}]{Opérations logiques}
\begin{itemize}
\item Trois opérations logiques de base permettent d'agir sur des variables binaires, de les combiner et de formuler des expressions logiques à partir d'elles.
\end{itemize}
\end{frame}


\begin{frame}[label={sec:org19426e6}]{Opérations logiques \ldots{} 1}
\begin{enumerate}
\item ET: cette opération est représentée (comme la multiplication) par
un point central ou par l'absence de signe d'opérateur entre les
arguments.  Par exemple, \(x \cdot y\) ou \(x y\).  La valeur de
l'expression est 1 si et seulement si toutes les variables ont la
valeur 1. Sinon, la valeur est 0.
\end{enumerate}
\end{frame}

\begin{frame}[label={sec:org9798199}]{Opérations logiques \ldots{} 2}
\begin{enumerate}
\item OU: cette opération est représentée (comme l'addition) par un signe
+. Par exemple, \(x + y\). La valeur de l'expression est 1 si au
moins une des variables a la valeur 1. Si aucune des variables ne
vaut 1, la valeur de l'expression est 0.
\end{enumerate}
\end{frame}

\begin{frame}[label={sec:org5926c57}]{Opérations logiques \ldots{} 3}
\begin{enumerate}
\item NON: cette opération est représentée par un prime, par
exemple \(x^\prime\), ou par une barre au-dessus de la variable,
\(\overline{x}\).  L'opération NON renverse la valeur binaire de
son argument: si \(x =0\) alors \(x^ \prime = 1\); si \(x =1\)
alors \(x^ \prime = 0\). Cette opération de négation, est aussi
appelée complément, car complémenter une valeur binaire revient à
faire basculer sa valeur.
\end{enumerate}
\end{frame}

\begin{frame}[label={sec:orga74d6a9}]{Expression logique}
\begin{itemize}
\item Une expression logique combine des variables logiques et des opérations et peut donc assumer une valeur binaire logique.

\item Cette valeur logique peut être assignée à une autre variable, en créant ainsi une équation logique.

\item Par exemple, \(z = x \cdot y\) signifie que \(z\) assume la valeur de l'expression \(x \cdot y\).

\item À partir des valeurs logiques des variables (entrées) \(x\) et \(y\), on peut donc déterminer la valeur logique de la sortie \(z\).
\end{itemize}
\end{frame}

\begin{frame}[label={sec:orge9ea3c2}]{Tableaux de vérité}
\begin{itemize}
\item On peut décrire la valeur logique d'une variable de sortie en fonction des valeurs possibles des variables d'entrée au moyen d'un tableau de vérité.

\item Dans un tel tableau, il y a une ligne pour chaque combinaison possible des valeurs d'entrée et, sur chaque ligne, on indique la valeur de sortie correspondante.

\item C'est en quelque sorte une description en extension de la valeur de l'expression de sortie.
\end{itemize}

Voici par exemple les tableaux de vérité pour les opérations de base.
\end{frame}

\begin{frame}[label={sec:org8678d16}]{Opération ET}
\begin{center}
\begin{tabular}{rrlr}
\(x\) & \(y\) &  & \(x \cdot y\)\\[0pt]
\hline
0 & 0 &  & 0\\[0pt]
0 & 1 &  & 0\\[0pt]
1 & 0 &  & 0\\[0pt]
1 & 1 &  & 1\\[0pt]
\end{tabular}
\end{center}
\end{frame}

\begin{frame}[label={sec:org0b2de98}]{Opération OU}
\begin{center}
\begin{tabular}{rrlr}
\(x\) & \(y\) &  & \(x + y\)\\[0pt]
\hline
0 & 0 &  & 0\\[0pt]
0 & 1 &  & 1\\[0pt]
1 & 0 &  & 1\\[0pt]
1 & 1 &  & 1\\[0pt]
\end{tabular}
\end{center}
\end{frame}

\begin{frame}[label={sec:org1c92b6a}]{Opération complément}
\begin{center}
\begin{tabular}{rlr}
\(x\) &  & \(x^{\prime}\)\\[0pt]
\hline
0 &  & 1\\[0pt]
1 &  & 0\\[0pt]
\end{tabular}
\end{center}
\end{frame}

\begin{frame}[label={sec:org629db48}]{Formalisme mathématique}
Un formalisme mathématique, élaboré bien avant l'avènement des circuits électroniques numériques, permet de formuler, analyser et simplifier les expressions de la logique binaire. Il s'agit de l'algèbre de Boole.

\begin{block}{Définitions}
\begin{itemize}
\item Une algèbre est un système mathématique, défini pour un ensemble d'éléments auxquels sont associés un ensemble d'opérateurs et qui respecte un jeu d'axiomes ou postulats.
\end{itemize}

Une algèbre nécessite donc:

\begin{enumerate}
\item Un ensemble \(S\) d'éléments

\item Des opérateurs: \(\cdot\), \(\star\), \(+\)
\end{enumerate}
\end{block}
\end{frame}

\begin{frame}[label={sec:org5d6b079}]{Formalisme mathématique \ldots{} 2}
\begin{enumerate}
\setcounter{enumi}{2}
\item L'application des opérateurs aux différents éléments doit respecter un certain nombre de propriétés appelées postulats, comme:
\end{enumerate}
\begin{itemize}
\item Fermeture

\item Associativité

\item Commutativité

\item Existence d'élément identité

\item Existence d'élément inverse

\item Distributivité
\end{itemize}
\end{frame}

\begin{frame}[label={sec:org00d9a20}]{Formalisme mathématique \ldots{} 3}
\begin{itemize}
\item Selon le choix des postulats, on arrive à définir différents types de systèmes algébriques.

\item Par exemple, les nombres réels qui nous sont familiers constituent un système algébrique d'un type appelé \alert{corps}.
\end{itemize}
\end{frame}

\begin{frame}[label={sec:orgd864595}]{Algèbre de Boole}
Une algèbre de Boole est un type de système algébrique défini sur un
ensemble \(B\), muni de deux opérateurs dénotés \(+\) et \(\cdot\), et qui
respecte les postulats suivants (postulats de Huntington):

Note: certains postulats viennent en paires; nous les distinguons ici au
moyen d'étiquettes \(\spadesuit\) ou \(\heartsuit\).
\end{frame}

\begin{frame}[label={sec:orgb57287d}]{Postulats}
\begin{enumerate}
\item Fermeture: tout résultat d'une opération sur un élément de
l'ensemble donne un élément de l'ensemble.

\begin{enumerate}
\item \(\spadesuit\) Fermeture par rapport à \(+\).

\item \(\heartsuit\) Fermeture par rapport à \(\cdot\).
\end{enumerate}

\item Éléments identité

\begin{enumerate}
\item \(\spadesuit\) Élément identité de \(+\), noté 0: on a \(x + 0 = 0 + x = x\).

\item \(\heartsuit\) Élément identité de \(\cdot\), noté 1: on a \(x \cdot 1 = 1 \cdot x = x\).
\end{enumerate}

\item Commutativité

\begin{enumerate}
\item \(\spadesuit\) Commutativité par rapport à \(+\): on a \(x + y = y + x\).

\item \(\heartsuit\) Commutativité par rapport à \(\cdot\): on a \(x \cdot y = y
        \cdot x\).
\end{enumerate}
\end{enumerate}
\end{frame}

\begin{frame}[label={sec:org82b7811}]{Postulats \ldots{} 2}
\begin{enumerate}
\setcounter{enumi}{3}
\item Distributivité

\begin{enumerate}
\item \(\spadesuit\) \(\cdot\) est distributif sur \(+\): on a \(x \cdot (y + z)= (x \cdot y) +
        (x \cdot z)\).

\item \(\heartsuit\) \(+\) est distributif sur \(\cdot\): on a \(x + (y \cdot z)= (x + y) \cdot
        (x + z)\).
\end{enumerate}

\item Pour chaque élément \(x \in B\), il existe un élément
\(x^{\prime} \in B\) (appelé complément de \(x\)) tel que

\begin{enumerate}
\item \(\spadesuit\) \(x + x^{\prime} = 1\).

\item \(\heartsuit\) \(x \cdot x^{\prime} = 0\).
\end{enumerate}

\item Il existe au moins deux éléments \(x, y \in B\) tels que \(x \neq y\).
\end{enumerate}
\end{frame}

\begin{frame}[label={sec:orga4f4e9e}]{Remarques}
Observons des différences entre une algèbre de Boole et le corps des réels:

\begin{enumerate}
\item Il n'y a pas de loi d'associativité dans les postulats. On peut en démontrer une, cependant.

\item L'opération \(+\) est distributive sur \(\cdot\).

\item Il n'y a pas d'inverse multiplicatif ni d'inverse additif, on ne peut donc pas faire de soustraction ou de division.

\item Il y a un concept de complément.

\item L'ensemble d'éléments est différent. Nous utiliserons pour notre part l'ensemble \(B: \{0, 1 \}\) pour notre algèbre de Boole.
\end{enumerate}
\end{frame}

\begin{frame}[label={sec:org1dd3424}]{Algèbre de Boole à deux valeurs}
\begin{columns}
\begin{column}{0.48\columnwidth}
\begin{block}{}
L'ensemble de définition: \(B : \{0, 1 \}\).

Opérateur \(\cdot\)

\begin{center}
\begin{tabular}{rrlr}
\(x\) & \(y\) &  & \(x \cdot y\)\\[0pt]
\hline
0 & 0 &  & 0\\[0pt]
0 & 1 &  & 0\\[0pt]
1 & 0 &  & 0\\[0pt]
1 & 1 &  & 1\\[0pt]
\end{tabular}
\end{center}
\end{block}
\end{column}


\begin{column}{0.48\columnwidth}
\begin{block}{}
Opérateur \(+\)

\begin{center}
\begin{tabular}{rrlr}
\(x\) & \(y\) &  & \(x + y\)\\[0pt]
\hline
0 & 0 &  & 0\\[0pt]
0 & 1 &  & 1\\[0pt]
1 & 0 &  & 1\\[0pt]
1 & 1 &  & 1\\[0pt]
\end{tabular}
\end{center}

Règle de complémentation

\begin{center}
\begin{tabular}{rlr}
\(x\) &  & \(x^{\prime}\)\\[0pt]
\hline
0 &  & 1\\[0pt]
1 &  & 0\\[0pt]
\end{tabular}
\end{center}
\end{block}
\end{column}
\end{columns}
\end{frame}


\begin{frame}[label={sec:orgeadb8d6}]{Vérification des postulats}
\begin{enumerate}
\item La fermeture est évidente (en regardant les tableaux des opérations).

\item En observant les tableaux de vérité, on constate que

\begin{enumerate}
\item \(0 + 0 = 0\), \(0 + 1 = 1 + 0 = 1\)

\item \(1 \cdot 1 = 1\), \(0 \cdot 1 = 1 \cdot 0 = 0\)
\end{enumerate}

ce qui définit les deux éléments identité: 0 pour \(+\) et 1 pour  \(\cdot\).

\item La commutativité des lois est évidente: les tableaux sont
symétriques.

\item Les lois de distributivité se démontrent aisément en établissant des
tableaux de vérité pour les différentes valeurs de \(x, y\) et \(z\).

\item Par le tableau de complément, on vérifie que

\begin{enumerate}
\item \(x + x^{\prime} = 1\), car \(0 + 0^{\prime} = 0 + 1 = 1\) et \(1 +
        1^{\prime} = 1+ 0 = 1\)

\item \(x \cdot x^{\prime} = 0\) car \(0 \cdot 0^{\prime} = 0 \cdot 1 =
        0\) et \(1 \cdot 1^{\prime} = 1 \cdot 0 = 0\).
\end{enumerate}

\item Le postulat 6 est vérifié car il y a deux éléments distincts: 0 et 1.
\end{enumerate}
\end{frame}
\end{document}